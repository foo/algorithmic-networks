\subsection{Contributions on static mapping}

This paper initiates the formal study of data-locality and replica aware virtual network embedding problems in datacenters.
In particular, we decompose the general optimization problem into its fundamental aspects, such as
assignment of chunks, replica selection, and flexible virtual machine
placement, and answer questions such as:
\begin{enumerate}
\item Which chunks to assign to which virtual machine?

\item How to exploit redundancy and select good replicas?

\item How to efficiently embed virtual machines and their inter-connecting network?

\item Can the chunk assignment, replica selection and virtual machine embedding problems be jointly optimized, in polynomial time?
\end{enumerate}

We draw a complete picture of the problem space: We show that
even problem variants exhibiting multiple degrees of freedom in terms of
replica selection and embedding,
can be solved optimally in polynomial time, and we present several efficient
algorithms accordingly. However, we also prove limitations in terms of
computational tractability, by providing reductions from 3-D matching
and Boolean satisfiability ($\SAT$). Interestingly,
while it is well-known that (unsplittable) multi-commodity flow
problems are NP-hard in capacitated networks, our hardness results also hold in \emph{uncapacitated}
networks; moreover, we show that NP-hard problems already arise in small-diameter networks (as they are
widely used today~\cite{fattree}).

\subsection{Contribution on dynamic mapping}


This paper introduces the online Balanced RePartitioning problem (BRP),
a fundamental \emph{dynamic} variant of the classic graph clustering problem. 
We show that BRP features some interesting connections to other well-known
online graph problems. For $\ell=2$, BRP is able to simulate online paging problem
and for for $k=2$, BRP is a~novel online version of maximum matching.
We consider deterministic algorithms and make the following technical
contributions:

\begin{description}

\item[Algorithms for General Variant:]
For the non-augmented variant, in \ref{sec:upper}, we first present a~simple
$O(k^2 \cdot \ell^2)$-competitive algorithm. Our main technical contribution
is an $O((1+1/\eps) \cdot k \log{k})$-competitive deterministic algorithm
$\CREP$ for a setting with $(2+\eps)$-augmentation (\ref{sec:crep}).
We emphasize that this bound does not depend on~$\ell$. This is interesting,
as in many application domains of this problem, $k$ is small: for example, in
our motivating virtual machine collocation problem, a server typically hosts
only a small number of virtual machines (e.g., related to the constant number
of cores on the server).

\item[Algorithms for Online Rematching:]
For the special case of online rematching ($k=2$, but arbitrary~$\ell$), in
\ref{sec:k-two}, we prove that a variant of a greedy algorithm is
7-competitive. We also demonstrate a lower bound of 3 for any deterministic
algorithm.

\item[Lower Bounds:]
By a reduction to online paging, in \ref{sec:paging}, we show that
for two clusters no deterministic algorithm can obtain a better bound than
$k-1$. While this shows an~interesting link between BRP and paging, in
\ref{sec:lower-bounds}, we present a stronger bound. Namely, we
show that for $\ell \geq 2$ clusters, no deterministic algorithm can beat the
bound of $k$ even with an~arbitrary amount of augmentation, as~long~as~the
algorithm cannot keep all nodes in a~single cluster. In contrast, online
paging is known to~become constant-competitive with constant
augmentation~\cite{SleTar85}.

\end{description}




\subsection{Contributions on memory management in network devices}

\todo{Mixed with organization of the paper}

We initiate the study of a natural new caching with bypassing problem which
allows to account for tree-dependencies among items. The problem finds
immediate applications, e.g., in IP routing and software-defined networking
(see \lref[Section]{sec:motivation}).

In particular, we consider the online tree caching problem within the resource
augmentation paradigm: we assume that cache sizes of the online algorithm
($\kALG$)  and the optimal offline algorithm ($\kOPT$) may differ. We assume
$\kALG \geq \kOPT$ and let $R = \kALG/(\kALG-\kOPT+1)$.

In \lref[Section]{sec:algo}, we present an elegant deterministic online
algorithm~\ALG for this problem. While our algorithm is simple, its analysis
presented in \lref[Section]{sec:analysis} requires several non-trivial
insights into the problem. In particular, we rigorously prove that \ALG is
$O(h(T) \cdot R)$-competitive, where $h(T)$ is the height of tree~$T$. That
is, we show that there exists a constant~$\beta$, such that $\ALG(I) \leq
O(h(T) \cdot R) \cdot \OPT(I) + \beta$ for any input $I$. Note that this
result is optimal up to the factor~$O(h(T))$: in
\lref[Appendix]{sec:lower-bound-on-the-problem}, we show that the lower
bound~$R$ for the paging problem~\cite{competitive-analysis} implies an
$\Omega(R)$ lower bound for our problem for any $\alpha \geq 1$. Finally, in
\lref[Section]{sec:implementing_counters}, we show that \ALG can be
implemented efficiently.

