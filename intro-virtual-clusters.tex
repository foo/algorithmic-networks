
\begin{enumerate}
  \item Physical datacenter architecture and physical clusters
  \item Sharing datacenter resources and isolation requirements
\end{enumerate}

\url{https://technet.microsoft.com/en-us/library/hh965746.aspx}

\subsubsection{VC abstract}

Virtualized datacenters offer great flexibilities in terms of resource allocation. In particular, by
decoupling applications from the constraints of the underlying infrastructure, virtualization
supports an optimized mapping of virtual machines as well as their interconnecting network
 (the so-called \emph{virtual cluster})
to their
physical counterparts: a graph embedding problem.

However, existing virtual cluster embedding algorithms often ignore a crucial dimension of the problem, namely \emph{data locality}:
the input to a cloud application such as MapReduce is typically stored in a distributed,
and sometimes redundant, file system. Since moving
data is costly, an embedding algorithm should be data locality aware,
and allocate computational resources close to the data; in case of redundant storage, the algorithm should also optimize the \emph{replica selection}.

This paper initiates the algorithmic study of data locality aware virtual cluster embeddings
on datacenter topologies.
We
show that
despite the multiple degrees of freedom in terms of embedding, replica selection and assignment,
many problems can be
solved efficiently. We also highlight the limitations of such optimizations,
by presenting several NP-hardness proofs; interestingly,
our hardness results also hold in uncapacitated networks of small diameter.

\subsubsection{VC practical motivations}

Our model is motivated by batch-processing applications such as MapReduce.
Such applications use multiple virtual machines to
process data, often redundantly stored in a distributed file system implemented
by multiple servers~\cite{local-schedule-1,mapreduce}.
Datacenter networks are typically organized as fat-trees, with servers are
located at the tree leaves and inner nodes being switches or routers.
Given the amount of multiplexing over the mesh of links
and the availability of multi-path routing protocol, e.g.~ECMP, the redundant
links can be considered as a single aggregate link for bandwidth
reservations~\cite{oktopus,infocom16,ccr15emb,proteus}.

During execution, batch-processing applications typically cycle through different phases,
most prominently, a map phase and a reduce phase; between the two phases,
a shuffling operation is performed, a phase where the results from the mappers
are communicated to the reducers. Since the shuffling phase can constitute a
non-negligible part of the overall runtime~\cite{orchestra},
and since concurrent network transmissions can introduce interference and
performance unpredictability~\cite{amazonbw}, it is important
to provide explicit minimal bandwidth guarantees~\cite{talk-about}.
In particular, we model the virtual network connecting the virtual machines
as a virtual cluster~\cite{oktopus,talk-about,proteus};
however, we extend this model with a notion of data-locality.
In particular, we distinguish between the bandwidth needed between the assigned chunk
and virtual machine ($\CostTrans$) and the bandwidth needed between
two virtual machines ($\CostCom$). 


\subsubsection{OBR abstract}

This paper initiates the study of the classic balanced graph partitioning
problem from an online perspective: Given an~arbitrary sequence of pairwise
communication requests between~$n$ nodes, with patterns that may change over
time, the objective is~to~service these requests efficiently by partitioning
the nodes into~$\ell$ clusters, each of size~$k$, such that frequently
communicating nodes are located in the same cluster. The partitioning can be
updated dynamically by \emph{migrating} nodes between clusters. The goal is to
devise online algorithms which jointly minimize the amount of inter-cluster
communication and migration cost.

The problem features interesting connections to other well-known online
problems. For example, scenarios with~$\ell=2$ generalize online paging, and
scenarios with~$k=2$ constitute a~novel online variant of maximum matching. We
present several lower bounds and algorithms for settings both with and without
cluster-size augmentation. In particular, we prove that any deterministic
online algorithm has a competitive ratio of at least~$k$, even with
\emph{significant} augmentation. Our main algorithmic contributions are
an~$O(k \log{k})$-competitive deterministic algorithm for the general setting
with constant augmentation, and a constant competitive algorithm for the
maximum matching variant.


\subsubsection{OBR practical motivations}


There are many applications to the dynamic graph clustering problem.
To give just one example, we consider server virtualization in
datacenters. Distributed cloud applications, including batch processing
applications such as MapReduce, streaming applications such as Apache Flink or
Apache Spark, and scale-out databases and key-value stores such as Cassandra,
generate a significant amount of network traffic and a considerable fraction
of their runtime is due to network acti\-vi\-ty~\cite{MogPop12}. For example,
traces of jobs from a Facebook cluster reveal that network transfers on
average account for 33\% of the execution time~\cite{ChZMJS11}. In such
applications, it is desirable that frequently communicating virtual machines
are \emph{collocated}, i.e., mapped to the same physical server, since
communication across the network (i.e., inter-server communication) induces
network load and latency. However, migrating virtual machines between servers
also comes at a price: the state transfer is bandwidth intensive, and may even
lead to short service interruptions. Therefore the goal is to design online
algorithms that find a good trade-off between the inter-server communication
cost and the migration cost.


