\documentclass[a4paper]{article}

\usepackage{amsfonts,amssymb,amsmath}
\usepackage{graphicx}
\usepackage[ruled,vlined,commentsnumbered]{algorithm2e}
\usepackage{hyperref}
\usepackage{multirow}
\usepackage[shortlabels]{enumitem}
\usepackage[utf8]{inputenc}
\usepackage{comment}
\usepackage{subfig}

\newcommand{\todo}[1]{\noindent\colorbox{red}{todo: #1}\margcomm{todo}} 
\newcommand{\marcin}[1]{\color{red} Marcin: #1\color{black}\margcomm{marcin}}
\newcommand{\maciek}[1]{\textcolor{blue}{(Maciek: #1)}\margcomm{maciek}}

%%%%%%%%%%%%%%%%%%%%%%%%%%%%%%%%%%%%%%%%%%%%%%%%%%%%%%%%%%%%%%%%%%%%%%%%%%%%%

\title{Algorithmic aspects of contemporary networks}

\author{Maciej Pacut}


\begin{document}
\maketitle

\section{Introduction}
\begin{enumerate}
  \item new challenges in networking
  \item SDN
\end{enumerate}

\subsection{Organization of the paper}

\section{Mapping virtual networks}

\subsection{Virtual networks with static topology}

\subsubsection{Introduction, practical motivations and related work}
Abstract/introduction compilation.
\begin{enumerate}
  \item The basic problem definition and its applications (with data locality and choice of replica)
  \item The general problem of embedding
  \item The substrate network description (the object to embed into)
  \item The virtual cluster description (the object to embed)
  \item The need of bandwidth reservations
  \item Practical motivations of Replica Selection, Multiple Assignment, Flexible Placement and Bandwidth Capacities
  \item No Node Interconnect $\rightarrow$ simplier problem
\end{enumerate}
Practical motivations ends here (+ possibly in the conclusions chapter)

\subsubsection{Model}
\begin{enumerate}
  \item Formal definition of the problem of embedding a virtual cluster in the substrate network
  \item The objective function
  \item Problem decomposition.
  \item Variant introduction: Replica Selection
  \item Variant introduction: Multiple Assignment
  \item Variant introduction: Flexible Placement
  \item Variant introduction: Node Interconnect
  \item Variant introduction: Bandwidth Constraints
\end{enumerate}
\subsubsection{Contributions/results}
\begin{enumerate}
  \item Short description of the results.
  \item The main question of the section: polynomially tractable or NP-complete
  \item P-time $\rightarrow$ minimize running time.
  \item NP-complete $\rightarrow$ how restricted version is still NP-complete
\end{enumerate}

\subsubsection{Polynomial-time algorithms intro}
\subsubsection{Matching-based polynomial-time algorithms}
\subsubsection{Flow-based polynomial-time algorithms}
\subsubsection{Dynamic programming-based polynomial-time algorithms}
\subsubsection{Note on degenerated problems}
\subsubsection{NP-hardness intro}
\begin{enumerate}
  \item The sketch of reductions and generalizations that applies to remainder of the problems (without polynomial-time algorithms)
  \item The 3D Multiple Assignment Problem + references to its NP-hardness
\end{enumerate}

\subsubsection{NP-hardness result for Multiple Assignment with at least 3 replicas}
\subsubsection{NP-hardness result for Node Interconnect with at least 3 replicas}

\subsubsection{Intro to restricted version of 2 replicas}

\subsubsection{NP-hardness result for Multiple Assignment with at most 2 replicas}
\subsubsection{NP-hardness result for Node Interconnect with at most replicas}

\subsubsection{Conclusions}

\subsection{Virtual networks with dynamic topology}

\subsubsection{Online analysis introduction?}

\subsubsection{Introduction, practical motivations and related work}

Abstract/introduction compilation.

\begin{enumerate}
  \item Make a clear connection to embedding from the previous chapter.
  \item Dynamic = discovering the structure of the object to embed in online fashion.
  \item Similar objective function: bandwidth reservation upfront and charging for the communication
  \item Virtual cluster = the worst case bandwidth reservation (complete graph) with KNOWN size; Virtual Cluster $\rightarrow$ data locality and replica selection part of the problem
  \item The extension of the model: migrations that take bandwith reservation (or communication cost) of certain size $\alpha$
\end{enumerate}

\subsubsection{Model}
\subsubsection{Results}
\subsubsection{Conclusions}


\section{Optimizing routing tables}

\subsubsection{Introduction}
\subsubsection{Practical motivations}
\subsubsection{Related work}
\subsubsection{Model}
\subsubsection{Results}
\subsubsection{Conclusions}


\section{Conclusions}

\end{document}
